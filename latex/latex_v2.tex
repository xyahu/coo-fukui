\documentclass[twocolumn]{article}
\usepackage[utf8]{inputenc}
\usepackage[margin=1in]{geometry}  % 调整页面边距
\usepackage{titling}               % 控制标题位置
\usepackage{booktabs}              % 表格宏包
\usepackage{amssymb}
\usepackage{ctex}                  % 中文支持包
\usepackage{authblk}			   % 多作者
% \usepackage{xeCJK}
\setlength{\droptitle}{-3.0cm}     % 将标题上移3.0厘米

\title{Medical Image Classification using Support Vector Machine}
\author[1] {Your Name}
\author[1] {Teacher Author}
% \author[1,3]{Third Author}
\affil[1]{University of Fukui, 3-9-1 Bunkyo, Fukui, 910-0019, Japan}
% \affil[2]{Department of Electrical Engineering, University of Y}
% \affil[3]{Research Institute, Company Z}
\date{2024.11.30}

\begin{document}

\twocolumn[
	\maketitle

	\begin{center}
		\begin{abstract}
			This is Abstract chapter.
		\end{abstract}
	\end{center}
	\vspace{0.5cm}  % 调整摘要与正文之间的间距
]

\section{Introduce}
支持向量机(SVM)是一种强大的监督式学习算法,主要用于分类和回归任务。它通过在特征空间中寻找最大间隔超平面来区分不同类别,有效提高模型的泛化能力。

\section{Related Works}
SVM很早应用于文本分类任务\cite{hearst_support_1998},如垃圾邮件检测和网页分类。在人脸识别、手写识别\cite{bahlmann_online_2002}和医学图像分析\cite{gautam_investigation_2021}等领域,SVM由于其高效的分类能力被广泛应用。支持向量机的开发和使用历史是机器学习领域中一个成功的案例,展示了理论研究如何转化为实际应用的工具。

\section{Method}
支持向量机(SVM)用于分类任务的目标是找到一个决策边界,即一个可以最大化地分隔不同类别数据点的超平面。超平面可以用以下等式表达:

\[
	f(x) = \mathbf{W}^\mathbb{T} x + b
\]

其中,\(f(x)\) 是模型的预测输出,输出一个实数值,表示样本 \(x_i\) 落在特定类别的置信度。\( \mathbf{W} \) 是超平面的法向量,\( b \) 是偏置项,\( x \) 是输入的特征向量。

SVM通过解决一个优化问题来找到最优的\( \mathbf{W} \)和\( b \),该优化问题旨在最大化两个类别之间的边缘。SVM 通常使用合页损失(Hinge Loss)来训练分类器,这是一种鼓励找到具有最大边缘的决策边界的方法。合页损失函数定义为:
\[
	L(y_i, f(x_i)) = \max(0, 1 - y_i f(x_i))
\]
其中,\( f(x_i) = \mathbf{W}^\mathbb{T} x_i + b \) 是模型的预测输出,输出一个实数值,表示样本 \(x_i\) 落在特定类别的置信度。\( y_i \) 是实际的类标签,它的取值为 \{-1, 1\}。
SVM 的工作原理是构造一个超平面,该超平面不仅可以正确分类所有训练数据点,还能最大化类别间的间隔。本研究将构建一个最基本的支持向量机,使用合页损失优化该向量机来实现一个医学图像的分类器。

\section{Experiments}
本研究将SVM用于医学图像分类任务,旨在构建一个SVM,输入一个医学图像,判断该医学图像为PET图像还是CT图像。In this study, the Lung PET or CT scan data\cite{li_large-scale_2020} were powered by the National Cancer Institute Cancer Imagine Program (CIP).该数据集涵盖了355名受试者的肺部扫描图像,共计251135张扫描图。这些数据主要收集自2009年至2011年间,包括了每位受试者的性别、年龄、体重、吸烟史及癌症诊断分类信息。数据集中的所有扫描数据均以DICOM格式存储。本研究利用Windows操作系统中的MicroDicom软件处理这251135份扫描数据。数据集中的受试者按癌症类型进行标记:类型A代表腺癌,类型B代表小细胞癌,类型E代表大细胞癌,类型G代表鳞状细胞癌。在该数据集中,并非所有受试者的资料均包含PET扫描与CT扫描。因此,本研究筛选仅使用了被诊断为小细胞癌(B类)的38名受试者的扫描数据,这些数据包括PET扫描、多种CT扫描以及融合增强后的扫描图像。在这38名受试者中,仅有9人同时拥有PET扫描与CT扫描的数据,共计12930张扫描图像。通过精确筛选,将切片位置误差不超过0.2mm的PET/CT扫描定义为配对扫描数据,最终获取928张扫描图像。这464对PET/CT肺部扫描数据图像,供本研究使用。

% 插入三线表
\begin{table}[h]
	\centering
	\caption{Experimental Dataset Partition}
	\label{tab:dataset_partition}
	\begin{tabular}{cccc}
		\toprule
		Params count & 128×128 & 256×256 & 512×512 \\
		\midrule
		Channel=1    & 16385   & 65537   & 262145  \\
		Channel=2    & 32769   & 131073  & 524289  \\
		Channel=3    & 49153   & 196609  & 786433  \\
		\bottomrule
	\end{tabular}
\end{table}

% 插入三线表
\begin{table}[h]
	\centering
	\caption{Number of Parameters to be Optimized in SVM Decision Functions for Different Input Image Sizes}
	\label{tab:params_count}
	\begin{tabular}{cccc}
		\toprule
		Params count & 128×128 & 256×256 & 512×512 \\
		\midrule
		Channel=1    & 16385   & 65537   & 262145  \\
		Channel=2    & 32769   & 131073  & 524289  \\
		Channel=3    & 49153   & 196609  & 786433  \\
		\bottomrule
	\end{tabular}
\end{table}

表\ref{tab:params_count}中,展示了构建针对不同尺寸的输入图像,构建出的决策函数中需要优化的参数数量。我们将得到的464对PET/CT肺部扫描数据导出为256×256像素的RGB格式的PNG图像。因此,本文需要构建输入数据$x$为$256 \times 256 \times 3=196608$维度的列向量的决策函数,因此$\mathbf{W}^\mathbb{T}$应为196608的行向量,和一个偏置项$b$。





\section{Conclusion}
This is Conclusion chapter.

\bibliographystyle{unsrt}
\bibliography{svm}

\end{document}
